Teoretická část

\chapter{Úvod do problematiky analýzy spotřeby energie}

Energetická spotřeba představuje jeden z klíčových ukazatelů fungování moderní společnosti. Rostoucí podíl obnovitelných zdrojů, 
jejichž výroba je proměnlivá a obtížně predikovatelná, klade stále vyšší nároky na detailní sledování, ukládání a analýzu energetických dat. Právě schopnost porozumět spotřebě energie a efektivně ji řídit je nezbytným předpokladem pro úspěšnou transformaci energetických systémů směrem k udržitelnosti. Uvádí ve své knize \cite{Sterner2019}

\section{Význam a využití energetických dat}

Energetická data představují základní informační zdroj pro pochopení a řízení spotřeby energie
v různých kontextech, od jednotlivých domácností přes průmyslové podniky až po celé městské
aglomerace. Jejich detailní sběr a analýza umožňuje nejen sledovat aktuální stav, ale také
předvídat budoucí vývoj a přijímat kvalifikovaná rozhodnutí. 
V knize \cite{Sterner2019} je popsán a využití energetických dat, který lze dále charakterizovat v souvislosti s:

Optimalizací provozu
\begin{itemize}
    \item \textbf{Snížení nákladů:} detailní data o spotřebě umožňují identifikovat neefektivní procesy,
    nadměrné ztráty nebo energeticky náročné zařízení.

    \item \textbf{Zvýšení efektivity:} na základě analýzy lze navrhnout opatření, jako je lepší regulace
    vytápění, optimalizace výrobních cyklů nebo využívání levnějších tarifů.
\end{itemize}

Predikcí zatížení sítí
\begin{itemize}
    \item \textbf{Stabilita elektrické soustavy:} energetická data umožňují provozovatelům sítí předvídat
    špičky v~odběru a přizpůsobit výrobu či distribuci.
    \item \textbf{Řízení rizik:} včasná predikce zatížení pomáhá předcházet výpadkům nebo nutnosti
    využívat drahé záložní zdroje.
    
\end{itemize}
\newpage

Integrací obnovitelných zdrojů
\begin{itemize}
    \item \textbf{Vyrovnávání rozdílů mezi výrobou a spotřebou:} obnovitelné zdroje (solární, větrné)
    jsou proměnlivé, proto je nutné jejich produkci sladit s aktuální poptávkou.
    \item \textbf{Podpora akumulace:} energetická data pomáhají rozhodovat, kdy ukládat energii do baterií
    či jiných úložišť a kdy ji uvolnit.
\end{itemize}

Podporou rozhodování
\begin{itemize}
    \item \textbf{Domácnosti:} spotřebitelé mohou sledovat svou spotřebu v reálném čase a přizpůsobit
    chování (např. využívat spotřebiče v době levnější elektřiny).
    \item \textbf{Podniky:} firmy využívají data pro plánování výroby, investic do úsporných technologií
    nebo pro certifikace v oblasti udržitelnosti.
    \item \textbf{Městské celky a státy:} energetická data slouží jako podklad pro strategické
    rozhodování, od plánování infrastruktury po tvorbu energetických politik.

\end{itemize}

 Od autorů \cite{Sterner2019} se též uvádí, že energetická data jsou klíčovým prvkem moderní energetiky. Umožňují optimalizovat provoz, předvídat zatížení sítí, 
 integrovat obnovitelné zdroje a podporovat rozhodování na všech úrovních od domácností až po národní 
politiky. V jejich pojetí jsou data základním předpokladem pro efektivní fungování energetických úložišť a pro stabilitu celého systému.


\section{Měření spotřeb elektrické energie}

Techniky měření spotřeby energie se v posledních desetiletích výrazně změnily a tento vývoj souvisí s celkovou proměnou energetických systémů směrem k digitalizaci a chytrému řízení. 
Zatímco dříve používané indukční elektroměry dokázaly ukázat jen celkový součet spotřeby, dnešní digitální a chytré měřiče umožňují sledovat odběr mnohem detailněji, a to v téměř reálném čase. Navíc dokážou komunikovat obousměrně s distribuční sítí.




(vysvětlit obousměrnou komunikaci) 
což otevírá nové možnosti pro práci s energetickými daty. Moderní systémy tak nejen zpřesňují samotné měření, ale zároveň poskytují podklady pro pokročilé analýzy, předpovědi zatížení nebo efektivní zapojení obnovitelných zdrojů do sítě.

Vývoj měřicích technologií prošel velkou proměnou, od jednoduchých indukčních elektroměrů až po dnešní digitální a chytré systémy. Zatímco dříve přístroje poskytovaly jen souhrnný údaj o spotřebě, moderní měřiče dokážou sledovat odběr v~reálném čase, komunikovat s~distribuční sítí a stát se součástí inteligentních sítí. To znamená, že měření už není jen o samotném odečtu, ale o~získávání dat, která pomáhají lépe řídit provoz, předvídat zatížení a rozhodovat na různých úrovních energetického systému. 
Vývoj měření tak odráží širší změnu celé energetiky, a to od mechanických přístrojů až po dnešní chytré technologie, které mění i způsob práce s daty. Prakticky to lze vidět třeba v~domácnosti: zatímco dříve se spotřeba elektřiny zjistila jen jednou za měsíc při odečtu, dnes chytrý elektroměr nebo aplikace ukáže, kdy je odběr nejvyšší, například večer při vaření nebo při zapnuté pračce. Díky tomu může člověk lépe plánovat používání spotřebičů, snížit náklady a mít větší kontrolu nad energií. 
Současné přístroje se tak stávají aktivním prvkem inteligentních sítí. Umožňují detailní monitoring, obousměrnou komunikaci a pokročilé analytické využití, což má podle \cite{Synergy2006} zásadní význam pro dnešní praxi i budoucí směřování energetických systémů.
Moderní měření spotřeby energie už není jen o tom, co ukáže klasický elektroměr na konci měsíce. Dnes se používají chytré měřiče a senzory, které sbírají data průběžně a mnohem přesněji. Díky tomu je vidět, kdy a kde se energie spotřebovává, například zda domácnost nejvíc zatěžuje síť večer při vaření nebo ráno při zapínání spotřebičů. Moderní měření spotřeby energie navíc umí ukázat spotřebu jednotlivých zařízení, takže lze zjistit, kde vznikají zbytečné ztráty. Takto získaná data pak slouží jako základ pro různé analýzy.
Bez moderního měření by se nedaly využívat metody strojového učení, které dokážou předpovědět budoucí spotřebu. Tyto technologie pomáhají optimalizovat provoz nebo lépe zapojit obnovitelné zdroje do sítě. Pro běžného člověka mají moderní měření velký přínos. 
Díky chytrým zásuvkám a aplikacím je možné vidět, které spotřebiče berou nejvíc elektřiny a kdy je odběr nejvyšší. To pomáhá lépe plánovat používání zařízení, snížit náklady a mít větší kontrolu nad energií.
Moderní měření tak není jen technická novinka, ale praktický nástroj, který podporuje úspory a efektivnější využívání energie v domácnosti.

\chapter{Popis dat a jejich struktura}
\section{teorie o časových řadách} 
Než začneme popisovat samotná data, je vhodné nejprve vymezit, co přesně rozumíme pojmem časová řada. V literatuře se časová řada zpravidla definuje jako chronologicky uspořádaná posloupnost hodnot určitého statistického ukazatele, přičemž jednotlivé hodnoty zachycují vývoj sledovaného jevu v čase.

„\textit{
Časová řada je chronologicky uspořádaná posloupnost hodnot určitého statistického ukazatele. \\[6pt]
$y_t = f(t)$ \\[6pt]
$y_1, y_2, \ldots, y_n$, kde $t = 1, 2, \ldots, n$ \\[6pt]
$y$ = ukazatel \\ 
$t$ = časová proměnná \\ 
$n$ = počet členů řady \\[6pt]
Pomocí časových řad můžeme zkoumat dynamiku jevů v čase. Mají základní význam pro analýzu příčin, které na tyto jevy působily a ovlivňovaly jejich chování v minulosti, tak pro předvídání jejich budoucího vývoje.
}“ \cite{Dobrovolny2006}


\section{Data o spotřebě energie}
Data o spotřebě energie jsou číselné hodnoty uspořádané v čase, které tvoří časové řady a umožňují sledovat, analyzovat a předpovídat energetické chování jednotlivců, domácností či celých systémů. V oblasti energetiky se velmi často setkáváme s pojmem kilowatthodina (kWh), který představuje základní jednotku pro vyjádření spotřeby elektrické energie. Přestože se objevuje na každém vyúčtování, jeho skutečný význam nebývá pro běžného spotřebitele vždy zcela jasný.

Každý elektrický spotřebič je charakterizován svým příkonem, tedy výkonem, který udává množství energie potřebné pro jeho provoz. 
Tento výkon se vyjadřuje ve wattech (W) nebo kilowattech (kW). Pokud zařízení o příkonu 1 kW pracuje po dobu jedné hodiny, spotřebuje právě 1 kWh energie. Z toho vyplývá, že kilowatthodina propojuje výkon spotřebiče s časem jeho provozu a vytváří tak údaj, který je přímo využitelný pro měření spotřeby.

Data o spotřebě energie jsou proto zaznamenávána v kilowatthodinách a mají charakter časových řad, kdy je každá hodnota spojena s konkrétním časovým okamžikem. Díky tomu lze sledovat nejen celkovou spotřebu, ale i její průběh v čase, což umožňuje identifikovat energeticky náročné spotřebiče, analyzovat uživatelské chování a optimalizovat provoz domácnosti či podniku.
 Vedle kilowatthodiny se v energetice používají i větší jednotky, například megawatthodina (MWh), která odpovídá tisíci kilowatthodinám. Tyto jednotky se uplatňují zejména při hodnocení výroby elektráren nebo při sledování spotřeby na úrovni celých regionů \cite{CSU2023}.
Spotřeba energie vyjádřená v~kilowatthodinách (kWh) se může výrazně lišit podle typu budovy a jejího provozu. Každý objekt má odlišné potřeby, které se odrážejí v~množství naměřených dat. Pro ilustraci lze uvést příklady různých typů budov:

\begin{enumerate}
    \item \textbf{Domácnost}: lze sledovat spotřebu běžných spotřebičů, osvětlení nebo případně vytápění. Data mohou ukázat, jak se spotřeba mění v~průběhu dne či roku a jak ji ovlivňují návyky obyvatel.

    \item \textbf{Škola}: měří se spotřeba spojená s~provozem učeben, počítačových laboratoří, kuchyně nebo tělocvičny. Naměřená data mohou odhalit rozdíly mezi běžným provozem během vyučování a minimální spotřebou o~víkendech či prázdninách.

    \item \textbf{Hotel}: má specifickou strukturu spotřeby, protože se jedná o~budovu s~nepřetržitým provozem pokojů, kuchyně, prádelny nebo klimatizace. Data tak mohou ukázat, jak se spotřeba mění podle obsazenosti hotelu a sezóny.

    \item \textbf{Nemocnice}: zde je spotřeba energie komplexnější. Sledují se údaje například z~provozu diagnostických přístrojů, sterilizačních zařízení, laboratoří, ale také z~osvětlení, klimatizace a dalších podpůrných služeb. Naměřená data jsou zde velmi různorodá a ukazují vysokou náročnost nepřetržitého provozu.

\item \textbf{Kancelářská budova}
\end{enumerate}
Představme si domácnost, kde se sleduje denní spotřeba elektřiny po dobu jednoho týdne. V pondělí byla spotřeba 12 kWh, v úterý 15 kWh, ve středu 14 kWh, ve čtvrtek 18 kWh, v pátek 20 kWh, v sobotu 25 kWh a v neděli 22 kWh. Každá hodnota je přitom pevně spojena s konkrétním dnem, a dohromady tak tvoří časovou řadu.

Z těchto údajů je možné vyčíst, že spotřeba postupně roste směrem k víkendu, kdy je domácnost více využívána. Zároveň se ukazuje sezónní prvek, tedy že víkendové dny mají vyšší spotřebu než pracovní dny. Náhodné fluktuace jsou patrné, například ve středu, kdy byla spotřeba nižší než v úterý, což může souviset s tím, že rodina nebyla večer doma. Taková časová řada nám tedy neukazuje jen jednotlivá čísla, ale odhaluje vzorce chování: kdy se spotřebovává více energie, jak se mění návyky během týdne a kde by bylo možné hledat úspory.


\subsection{Elekroměry}
Elektroměr, někdy označovaný jako elektrické hodiny, je zařízení určené k měření spotřeby elektrické energie v domácnostech. Zatímco první modely připomínaly klasické hodiny s~ručičkami a číselníkem, postupně je nahradily moderní varianty, například statické či impulsní. Spotřeba se udává v kilowatthodinách (kWh) a dnešní elektroměry jsou mnohem přesnější než jejich předchůdci. Impulsní typ funguje na principu počítání impulsů odpovídajících jedné kWh, a kromě samotného měření dokáže zobrazit i další údaje, například maximální odběr nebo komunikovat s počítačem \cite{CEZ2021}.

Úloha elektroměru však není jen technická. Jeho údaje slouží jako základ pro vyúčtování elektřiny. 
Protože je umístěn na rozhraní mezi distributorem a odběratelem, umožňuje přesně určit, kolik energie bylo spotřebováno a jaká částka bude účtována. 
Cena elektřiny se přitom neodvíjí pouze od množství odebrané energie, ale i od dalších faktorů, které ovlivňují konečný účet. Elektroměry se používají nejen v domácnostech, ale také ve školách či podnicích, kde poskytují klíčové údaje pro vyúčtování i analýzu spotřeby. 
Moderní přístroje navíc umožňují sledovat průběh odběru v čase, což otevírá prostor pro efektivnější hospodaření s elektřinou. Rozdíly mezi jednotlivými typy měření jsou patrné zejména v tom, jak detailně dokáží spotřebu zaznamenat. V domácnostech se nejčastěji používá neprůběhové měření typu C, které ukazuje pouze celkový stav. U větších odběratelů, jako jsou školy nebo úřady, 
se naopak uplatňuje průběhové měření typu B. To sleduje spotřebu v pravidelných intervalech, od července 2024 konkrétně každých 15 minut. Tento způsob je povinný, např. pro odběrná místa  s fotovoltaikou, pro sdílení elektřiny nebo pro zákazníky s produkty navázanými na krátkodobé trhy. Chytré elektroměry, které průběhové měření umožňují, automaticky ukládají data a odesílají je distributorovi. Díky tomu je možné nejen přesně účtovat spotřebu, ale také přizpůsobit provoz domácnosti či instituce aktuálním tarifům \cite{ENERGIE,2025}.
\subsection{druhy Elektroměrů}
Princip elektroměru není ve skutečnosti příliš složitý. 
Zařízení pracuje jako integrátor, který dokáže zaznamenávat průběh spotřeby. Měření může probíhat buď přímo, kdy elektrický proud prochází samotným elektroměrem, nebo nepřímo, kdy se využívají měřicí transformátory obepínající přívodní fázové vodiče. Je také důležité zdůraznit, že odlišný způsob fungování najdeme u klasického mechanického indukčního elektroměru a u moderního elektronického impulsního typu \cite{CEZ2021}. V~současnosti se pro měření spotřeby energie používají tři základní typy elektroměrů, které se liší principem měření i~způsobem zobrazování odečtů. Jedná se o:
\begin{enumerate}
    \item \textbf{Elektrodynamické (indukční) elektroměry}

Indukční elektroměr patří mezi starší typy měřicích zařízení. Jeho činnost je založena na otáčení hliníkového kotouče, který se dostává do pohybu působením vířivých proudů. V~jádru elektroměru jsou navinuty napěťová a proudová cívka, jejichž elektromagnetické pole působí proti poli vířivých proudů a tím roztáčí kotouč. Na něj je napojeno mechanické počítadlo, které převádí počet otáček na hodnotu spotřeby, takže odběr elektřiny lze sledovat přímo v~reálném čase \cite{CEZ2021}.

Tento typ zařízení má své přednosti i slabiny. Mezi nevýhody patří nižší přesnost měření, závislost chyby na velikosti proudu a nemožnost automatizovat odečty. Naopak jeho výhodou je dlouhá životnost, odolnost vůči kolísání napětí či vysokonapěťovým impulsům, spolehlivost a nízká pořizovací cena. Právě díky těmto vlastnostem se indukční elektroměry stále využívají zejména ve venkovských oblastech nebo na chatách, kde hrozí nestabilita sítě či riziko zásahu blesku \cite{TypyElektromeruND}.
\begin{figure}[H]
    \centering
    \includegraphics[width=0.6\textwidth]{elektromer_1.png}
    \caption{Moderní elektroměr \cite{ET404}}.
    \label{fig:elektromer1}
\end{figure}

    \item \textbf{Elektronické elektroměry}

Elektronické elektroměry fungují na principu měření napětí a proudu, jejich vzájemného násobení a ukládání výsledků do paměti. Díky tomu dokáží zobrazit nejen spotřebovanou energii, ale i~další parametry přímo na displeji. Oproti starším indukčním typům nabízejí vyšší přesnost, možnost využívat více tarifů a snadné propojení s~automatizovanými systémy řízení či sběru dat. Jejich slabinou je kratší životnost, složitější opravy a vyšší citlivost na přepětí, přesto se staly běžným standardem a postupně nahradily mechanické modely \cite{TypyElektromeruND}.

Specifickou variantou je elektronický impulsní elektroměr, který převádí naměřené hodnoty na impulsy odpovídající spotřebované energii. Každý impuls představuje určitou konstantu, obvykle jednu kilowatthodinu. Uvnitř zařízení se nacházejí obvody zpracovávající činné i jalové\footnote{Jalové napětí je složka střídavého napětí spojená s indukčními a kapacitními prvky v obvodu, která se nepodílí na užitečném výkonu, ale způsobuje pouze výměnu energie mezi zdrojem a spotřebičem} složky napětí, které jsou vedeny přes převodník a zesilovač do násobičky. Tam dochází k jejich kombinaci a výsledkem je série impulsů zobrazovaných na displeji, díky nimž lze sledovat spotřebu energie v~reálném čase \cite{CEZ2021}.

\begin{figure}[H]
    \centering
    \includegraphics[width=0.6\textwidth]{elektromer_2.png}
    \caption{Elektronický elektroměr}
    \label{fig:elektronicky_elektromer}
\end{figure}

\item \textbf{Hybridní elektroměry}

Hybridní elektroměr představuje konstrukční kompromis, kdy samotné měření výkonu probíhá pomocí elektronických součástek, 
ale výsledky se zobrazují na klasickém mechanickém počítadle. Tento typ vznikl v~době, kdy byly elektronické komponenty drahé, a výrobci se snažili snížit cenu elektronických měřičů. Při provozu vydává charakteristické mechanické cvakání, což může být rušivé při instalaci v~domácnosti nebo bytě. Přestože jde o~méně úspěšné řešení s~řadou nedostatků, hybridní elektroměry se stále dají najít v~nabídce obchodů \cite{TypyElektromeruND}.

\begin{figure}[H]
    \centering
    \includegraphics[width=0.6\textwidth]{elektromer_3.png}
    \caption{Hybridní elektroměr.}
    \label{fig:hybridni_elektromer}
\end{figure}
\end{enumerate}

\subsection{Struktura popisu a sběru dat}

\begin{figure}[H]
    \centering
    \includegraphics[width=0.9\textwidth]{dataset.png}
    \caption{Výsledky z Visual Studia}
    \label{fig:vysledekVS2}
\end{figure}

vlastní zpracování autora


\begin{figure}[H]
    \centering
    \includegraphics[width=0.9\textwidth]{dataset 2.png}
    \caption{Výsledky z Visual Studia}
    \label{fig:vysledekVS}
\end{figure}
vlastní zpracování autora


Tyto data, se kterýma budu pracovat pocházejí ze stránky 
https://disk.portabo.cz/s/fXEC0sJRhU7ChrV 
a jsou to zmiňovaná reálná data odečtů elektrické energie z různých objektů u nichž je pouze číeslný faktor
v tomto datasetu se nachází přes 10 000 záznámů, které jsou aktualizované po 15 minutách. Pro ukázku jsem vložil 2 screenshoty z toho které ukazují tyto hodnoty:
údaj v ts je časový údaj (datum a čas) po 15 minutách. 
id numeric je identifikátor objektu (číslený identifikátor jednotlivých objektů bez konkrétního typu)
anoymized value je údaj zaznanemaného množství spotřebované energie)



\section{použití metod strojového učení pro analýzy časových řad}
Metody strojového učení se v posledních letech staly klíčovým nástrojem pro analýzu časových řad. Dokáží zachytit složité vzorce a závislosti, které tradiční statistické postupy často nedokáží postihnout. Díky nim lze nejen lépe porozumět minulému vývoji, ale také přesněji předpovídat budoucí hodnoty, např. v~oblasti energetiky či klimatických procesů.
\subsection{pojem časové řady a její charakteristiky}
Časové řady představují zvláštní typ dat, jejichž podstatou je sledování hodnot v čase. Na rozdíl 
od běžných statistických souborů, kde pořadí jednotlivých pozorování nehraje roli, zde časová posloupnost určuje význam i interpretaci. Každý záznam je výsledkem předchozího vývoje a současně ovlivňuje budoucí hodnoty, což z časových řad činí klíčový nástroj pro analýzu dynamických jevů,od ekonomických ukazatelů přes energetickou spotřebu až po demografické či klimatické procesy. 
 \cite{Prince2023} Tento rámec rozšiřuje tím, že časové řady označuje za typ sekvenčních dat, jejichž hlavní charakteristikou je vzájemná provázanost jednotlivých prvků. Právě tato závislost odlišuje časové řady od jiných datových struktur a vysvětluje, proč se staly středem zájmu moderních metod strojového učení.
Klasická statistická literatura vymezuje časové řady prostřednictvím jejich základních komponent: trendu, sezónnosti, cykličnosti a náhodné složky. Jak ukazují \cite{BoxJenkins1976}, identifikace těchto prvků je nezbytným předpokladem pro konstrukci vhodných predikčních modelů, zejména typu ARIMA. 
Podobně \cite{Hamilton1994} zdůrazňuje, že rozlišení systematických a náhodných složek časové řady umožňuje lépe porozumět dynamice sledovaného jevu a oddělit dlouhodobé trendy od krátkodobých fluktuací. Tento rámec tvoří východisko pro pokročilé modelování a predikci, které se v současnosti stále častěji opírá o metody strojového učení a hlubokých neuronových sítí.

\cite{Prince2023} dále zdůrazňuje, že časové řady jsou typickým příkladem sekvenčních dat, tedy dat, 
u nichž pořadí jednotlivých hodnot hraje zásadní roli. Na rozdíl od klasického statistického přístupu (Box–Jenkins, Hamilton), který pracuje s komponentami jako trend, sezónnost či cykličnost, Prince se soustředí na to, jak tyto závislosti mezi prvky zachytit pomocí metod hlubokého učení.



Rekurentní neuronové sítě (RNN) představují první architekturu schopnou pracovat se sekvenčními daty, tedy i s časovými řadami. Jejich princip spočívá v tom, že každý výstup závisí nejen na aktuálním vstupu, 
ale také na předchozím stavu sítě, což jim umožňuje uchovávat informaci o minulých hodnotách a využívat ji při predikci. \cite{prince 2023} upozorňuje, že RNN dokáží zachytit krátkodobé závislosti mezi jednotlivými prvky sekvence, avšak při práci s dlouhými sekvencemi trpí problémem vyhasínajících gradientů, což omezuje jejich schopnost uchovávat dlouhodobý kontext

Na tuto slabinu reagoval vývoj pokročilejších architektur, jako jsou Long Short-Term Memory (LSTM) a Gated Recurrent Units (GRU). \cite{Prince2023} vysvětluje, že tyto sítě zavádějí paměťové mechanismy (tzv. brány), které regulují tok informací a umožňují uchovávat dlouhodobé závislosti, například sezónní 
cykly nebo dlouhodobé trendy ve spotřebě energie. Díky tomu se LSTM a GRU staly standardem pro predikci časových řad.
Další zásadní posun představují modely typu Transformer.  \cite{Prince2023} zdůrazňuje, že tyto sítě opouštějí princip rekurence a místo toho využívají mechanismus pozornosti (attention), který umožňuje modelu dynamicky určovat váhu jednotlivých prvků sekvence. Transformery tak dokáží efektivně zachytit 
i velmi dlouhé závislosti, aniž by trpěly problémem vyhasínajících gradientů. Tato architektura se stala současným vrcholem v analýze sekvenčních dat a její využití se rychle rozšířilo z oblasti zpracování přirozeného jazyka do dalších domén, včetně predikce ekonomických ukazatelů, modelování energetické spotřeby či 
analýzy biologických signálů.
Vývoj metod hlubokého učení pro analýzu časových řad lze podle \cite{Prince2023} chápat jako postupné překonávání limitů jednotlivých architektur. RNN představovaly první krok, LSTM a GRU přinesly schopnost uchovávat dlouhodobý kontext a Transformery otevřely cestu k modelování komplexních vztahů v rozsáhlých 
sekvencích. Tento vývoj odráží snahu postupně překonávat omezení předchozích přístupů a vytvářet stále robustnější rámec pro analýzu sekvenčních dat.
Když se mluví o různých typech neuronových sítí, je vhodné si je představit na příkladu předpovědi spotřeby elektřiny. Jednoduché RNN fungují tak, že si pamatují pouze krátkou historii. Hodí se například 
pro odhad, jak bude zatížení sítě vypadat během několika následujících hodin, podobně jako když sledujeme, že večer lidé zapínají televizi a spotřeba krátce stoupne. LSTM sítě mají navíc paměťové mechanismy, díky čemuž dokážou pracovat s delšími časovými úseky. Umí tak zachytit sezónní vzorce, například že v zimě je spotřeba energie 
vyšší než v létě, nebo že o víkendech bývá jiná než ve všední dny. GRU fungují podobně jako LSTM, ale jejich  architektura je jednodušší a výpočetně úspornější, což je výhodné tam, kde je třeba zpracovávat velké množství dat 
ze senzorů v reálném čase.

Nejmodernější Transformery dokáží pracovat s velmi dlouhými časovými řadami a hledat v nich komplexní souvislosti.V energetice se používají například k odhadu zatížení celé distribuční sítě, a to nejen na několik hodin dopředu, 
ale i s ohledem na dlouhodobé trendy. To je zásadní například při plánování kapacity sítě v situacích, kdy se očekává nárůst počtu domácností s fotovoltaikami nebo elektromobily.

 LSTM a GRU představují zásadní krok k překonání omezení klasických rekurentních 
sítí. \cite{Raschka2019} pak ukazuje, jak lze tyto modely prakticky implementovat v prostředích jako TensorFlow nebo Keras. Zdůrazňuje, že jejich síla spočívá v tom, že dokážou rozhodnout, které informace si ponechat, které 
zahodit a které dále využít.To je klíčový předpoklad při práci s časovými řadami  ať už jde o finanční data,nebo měření ze senzorů v energetice.
Na rozdíl od tradičních statistických metod \cite{BoxJenkins1976, Hamilton1994}, které časové řady rozkládají na trend, sezónnost a náhodné složky, přístup strojového učení se nesnaží tyto části explicitně pojmenovat. Místo toho se model učí přímo ze samotných dat a hledá v nich vzorce automaticky. 
Tento přístup ilustruje posun od klasické statistiky k moderním metodám hlubokého učení, kde je důraz kladen na schopnost modelu samostatně odhalovat souvislosti.
\subsection{Popis perspektivních modelů strojového učení vzhledem k datům}
Raschka \cite{Raschka2019} ukazuje, že úspěch modelů strojového učení vždy závisí na tom, s jakými daty pracujeme. 
Klasické statistické metody, jako ty od Boxe a Jenkinse nebo Hamiltona, předpokládají, že časovou řadu lze rozdělit na části, například trend, sezónnost nebo náhodné výkyvy. Moderní neuronové sítě ale fungují jinak, kdy se 
vzorce učí přímo z dat, aniž by je musely předem rozkládat. To znamená, že modely jako RNN se hodí na krátké sekvence, například pokud chceme odhadnout, jak se bude vyvíjet spotřeba elektřiny během několika hodin. 
LSTM a GRU mají paměť, takže zvládnou pracovat i s delšími obdobími. Díky tomu dokážou zachytit sezónní vzorce, například že v zimě je spotřeba vyšší kvůli topení, zatímco v létě se více používají klimatizace. Transformery pak 
dokáží zpracovat obrovské množství dat najednou a hledat v nich složité souvislosti. V praxi se využívají například k odhadu zatížení celé distribuční sítě, kde je potřeba zohlednit nejen krátkodobé výkyvy, ale i dlouhodobé trendy, jako je postupný nárůst počtu elektromobilů.

Moderní architektury dokážou zachytit komplexní závislosti v datech, zatímco Raschka \cite{Raschka2019} se soustředí na to, jak je prakticky naprogramovat a použít. V energetice to znamená, 
že modely mohou pracovat nejen s okamžitými změnami spotřeby, ale i s dlouhodobými vzorci chování budov, například že kancelářské budovy mají jiný profil než nemocnice nebo školy. Díky tomu lze budovy klasifikovat podle jejich 
dynamických energetických profilů, nikoli jen podle statických údajů, jako je velikost nebo počet místností.

Sschopnost analyzovat velké množství senzorových dat je klíčová pro rozvoj chytrých energetických sítí. Pokud systém dokáže rozpoznat dlouhodobé vzorce spotřeby a reagovat na změny, 
může lépe řídit tok energie a předcházet přetížení. V praxi to znamená, že město může například lépe plánovat, kdy zapojit záložní zdroje, nebo jak optimalizovat využití obnovitelných zdrojů.
Volba modelu strojového učení vždy závisí na povaze dat. V energetice to znamená, že jiný přístup se hodí pro domácnost, jiný pro nemocnici nebo hotel. Jednodušší RNN zvládnou krátkodobé odhady, například 
předpovědět, že v domácnosti večer stoupne spotřeba, když se zapínají spotřebiče. LSTM a GRU dokážou pracovat s delšími časovými řadami a zachytit sezónní vzorce. To je užitečné například ve škole, kde je spotřeba výrazně 
vyšší během školního roku a nižší o prázdninách, nebo v nemocnici, kde se spotřeba mění podle ročních období. 
Tyto modely si „pamatují“ delší kontext, takže dokážou předvídat i cyklické chování. Transformery se uplatní tam, kde je potřeba zpracovat velké množství dat najednou a hledat složité souvislosti. Typickým příkladem je hotel, kde spotřeba závisí nejen na ročním období, ale i na obsazenosti pokojů, počasí nebo konání akcí. 
Transformer dokáže propojit všechny tyto faktory a nabídnout přesnější predikci. Pro statické údaje, jako je velikost budovy nebo typ vytápění, se hodí klasifikační algoritmy typu XGBoost nebo Random Forest. V praxi se 
často kombinuje analýza časových řad s těmito metadaty, například nemocnice má jiný profil než škola, i když obě vykazují sezónní cykly.Celkově tedy platí, že perspektivnost modelů není dána jen jejich architekturou, ale hlavně schopností přizpůsobit se konkrétním datům. V kontextu energetiky to znamená, že správná volba modelu je klíčová pro klasifikaci budov 
podle jejich dynamických energetických profilů, od domácností přes školy až po nemocnice a hotely. Díky tomu lze lépe plánovat spotřebu, předcházet přetížení sítě a optimalizovat provoz.